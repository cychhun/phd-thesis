\label{titlePageLayout}

% Méta-données du PDF / PDF meta-datas
\hypersetup{
    pdfauthor={Cyril Chhun},
    pdfsubject={Manuscrit de thèse de doctorat},
    pdftitle={Meta-Evaluation Methodology and Benchmark for Automatic Story Generation},
}

\thispagestyle{empty}

{\color{black} \hfill \vfill \tiny \ecodocnum}
\begin{textblock}{5}(0,0)
    \textblockcolour{black}
    %\vspace{10mm}
    \includegraphics [scale=0.95]{titlePage/pictures/bande}
    \vspace{300mm}
\end{textblock}


\begin{textblock}{1}(0.6,3)
    \Large{\rotatebox{90}{\color{white}{\textbf{NNT : \NNT}}}}
\end{textblock}
                            

\begin{textblock}{1}(\hpostt,\vpostt)
    \textblockcolour{white}
    \includegraphics[scale=1]{titlePage/pictures/\logoEtt.png} 
\end{textblock}

\begin{textblock}{1}(\hpos,\vpos)
    \textblockcolour{white}
    \includegraphics[scale=1]{titlePage/pictures/\logoEt.png}
\end{textblock}

%\vspace{6cm}
%% Texte
\begin{textblock}{10}(5.7,3)
    \textblockcolour{white}
    
    \color{black}
    %\begin{center}  
    \begin{flushright}
        \huge{\PhDTitle} \bigskip %% Titre de la thèse 
        \vfill
        \color{black} %% Couleur noire du reste du texte
        \normalsize {Thèse de doctorat de l'Institut Polytechnique de Paris} \\
        préparée à \PhDworkingplace \\ \bigskip
        \vfill
        École doctorale n$^{\circ}$\ecodocnum ~\ecodoctitle ~(\ecodocacro)  \\
        
        \small{Spécialité de doctorat: \PhDspeciality} \bigskip %% Spécialité 
        \vfill  
        \footnotesize{Thèse présentée et soutenue à \defenseplace, le \defensedate, par} \bigskip
        \vfill
        \Large{\textbf{\textsc{\PhDname}}} %% Nom du docteur
        \vfill
        %\bigskip
    \end{flushright}
    
    %\end{center}
    \color{black}
    %% Jury
    \begin{flushleft}
        
        \small Composition du Jury :
    \end{flushleft}
    %% Members of the jury

    \small
    %\begin{center}
    \newcolumntype{L}[1]{>{\raggedright\let\newline\\\arraybackslash\hspace{0pt}}m{#1}}
    \newcolumntype{R}[1]{>{\raggedleft\let\newline\\\arraybackslash\hspace{0pt}}lm{#1}}
    
    \label{jury} 																				%% Mettre à jour si des membres ont été ajoutés ou retirés / Update if members have been added or removed
    \begin{flushleft}
    \begin{tabular}{@{} L{9.5cm} R{4.5cm}}
        % \jurynameB  \\ \juryadressB & \juryroleB \\[5pt]
        \jurynameD \\ \juryadressD & \juryroleD \\[5pt]
        \jurynameC  \\ \juryadressC & \juryroleC \\[5pt]
        \jurynameA  \\ \juryadressA & \juryroleA \\[5pt]
        \jurynameE  \\ \juryadressE & \juryroleE \\[5pt]
        \jurynameF  \\ \juryadressF & \juryroleF \\[5pt]
        \jurynameG  \\ \juryadressG & \juryroleG \\[5pt]
        % \jurynameH  \\ \juryadressH & \juryroleH \\[5pt]
    \end{tabular} 
    \end{flushleft}   
    %\end{center}

\end{textblock}
%%%%%%%%%%%%%%%%%%%%%%%%%%%%%%%%%%%%%%%%
%           Liste des packages         %
%%%%%%%%%%%%%%%%%%%%%%%%%%%%%%%%%%%%%%%%

\usepackage{blindtext}
\usepackage{lipsum}
\usepackage[colorinlistoftodos]{todonotes}

%%%%%%%%%%%%%%%%%%%%%%%%%%%%%%%%%%%%%%%%%%%%%%%%%%%%%%%%%%%%%%%%%%%%%

%% Réglage des fontes et typo    
\usepackage[utf8]{inputenc}		% LaTeX, comprend les accents !
\usepackage[T1]{fontenc}

\usepackage[square,sort&compress]{natbib}		% Doit être chargé avant babel

% \usepackage[square,sort&compress,sectionbib]{natbib}		% Doit être chargé avant babel
% \usepackage{chapterbib}
% 	\renewcommand{\bibsection}{\section{References}}		% Met les références biblio dans un \section (au lieu de \section*)
		
\usepackage[main=english,french]{babel}
\usepackage{lmodern}
\usepackage{ae,aecompl}										% Utilisation des fontes vectorielles modernes
\usepackage{fourier}
\usepackage{helvet}

%%%%%%%%%%%%%%%%%%%%%%%%%%%%%%%%%%%%%%%%%%%%%%%%%%%%%%%%%%%%%%%%%%%%%
% Allure générale du document
\usepackage{emptypage}      % truly empty pages
\usepackage{enumerate}
\usepackage{enumitem}
\usepackage[section]{placeins}	% Place un FloatBarrier à chaque nouvelle section
\usepackage{epigraph}
\usepackage[font={small}]{caption}
\usepackage[nohints]{minitoc}		% Mini table des matières, en français
	\setcounter{minitocdepth}{2}	% Mini-toc détaillées (sections/sous-sections)
\usepackage{titlesec}		% Ajoute les Tables	des Matières/Figures/Tableaux à la table des matières
\usepackage[numbib]{tocbibind}		% Ajoute les Tables	des Matières/Figures/Tableaux à la table des matières

%%%%%%%%%%%%%%%%%%%%%%%%%%%%%%%%%%%%%%%%%%%%%%%%%%%%%%%%%%%%%%%%%%%%%
%% Maths                         
\usepackage{mathtools}			% Permet de taper des formules mathématiques
% \usepackage{amsmath}
\usepackage{amssymb}			% Permet d'utiliser des symboles mathématiques
\usepackage{amsfonts}			% Permet d'utiliser des polices mathématiques
\usepackage{nicefrac}			% Fractions 'inline'

%%%%%%%%%%%%%%%%%%%%%%%%%%%%%%%%%%%%%%%%%%%%%%%%%%%%%%%%%%%%%%%%%%%%%
%% Tableaux
\usepackage{multirow}
\usepackage{booktabs}
% \usepackage{colortbl}
% \usepackage{tabularx}
% \usepackage{threeparttable}
% \usepackage{etoolbox}

%%%%%%%%%%%%%%%%%%%%%%%%%%%%%%%%%%%%%%%%%%%%%%%%%%%%%%%%%%%%%%%%%%%%%
%% Eléments graphiques                    
\usepackage{graphicx}			% Permet l'inclusion d'images
% \usepackage{subcaption}
% \usepackage{pdfpages}
% \usepackage{rotating}
% \usepackage{pgfplots}
% 	\usepgfplotslibrary{groupplots}
% \usepackage{tikz}
% \usepackage{listofitems} % for \readlist to create arrays
% \usetikzlibrary{arrows.meta} % for arrow size
% % \usetikzlibrary{positioning}
	% \usetikzlibrary{backgrounds,automata}
	% \pgfplotsset{width=7cm,compat=1.3}
	% \tikzset{every picture/.style={execute at begin picture={
 %   		\shorthandoff{:;!?};}
	% }}
	% \pgfplotsset{every linear axis/.append style={
	% 	/pgf/number format/.cd,
	% 	use comma,
	% 	1000 sep={\,},
	% }}
% \usepackage{eso-pic}
\usepackage{import}
\usepackage{framed}

\usepackage[absolute,overlay]{textpos}      % titlePageLayout textblocks

%%%%%%%%%%%%%%%%%%%%%%%%%%%%%%%%%%%%%%%%%%%%%%%%%%%%%%%%%%%%%%%%%%%%%
%% Mise en forme du texte        
% \usepackage[version=3]{mhchem}	% Equations chimiques
% \usepackage{textcomp}
\usepackage{array}        % titlePageLayout
\usepackage{microtype}
% \usepackage{hyphenat}
\usepackage{xcolor}
\usepackage{pifont}
% \usepackage{framed}
\usepackage{tcolorbox}      % chapter abstracts
\tcbuselibrary{theorems}
\usepackage{multicol}       % titlePageLayout
\setlength{\columnseprule}{0pt}
\setlength\columnsep{10pt}

%%%%%%%%%%%%%%%%%%%%%%%%%%%%%%%%%%%%%%%%%%%%%%%%%%%%%%%%%%%%%%%%%%%%%
%% Navigation dans le document
\usepackage[pdftex,pdfborder={0 0 0},
			colorlinks=true,
			linkcolor=blue,
			citecolor=red,
			pagebackref=true,
			]{hyperref}	% Créera automatiquement les liens internes au PDF
					% Doit être chargé en dernier (Sauf exceptions ci-dessous)
			
%%%%%%%%%%%%%%%%%%%%%%%%%%%%%%%%%%%%%%%%%%%%%%%%%%%%%%%%%%%%%%%%%%%%%
%% Packages qui doivent être chargés APRES hyperref	   
% \usepackage[top=2.5cm, bottom=2cm, left=3cm, right=2.5cm,
% 			headheight=15pt]{geometry}
\usepackage{geometry}
% \usepackage{lscape}
% \usepackage{rotating}
\usepackage{pdflscape}
\usepackage{afterpage}

\usepackage{fancyhdr}			% En-tête et pied de page. Doit être placé APRES geometry
	\pagestyle{fancy}		% Indique que le style de la page sera justement fancy
	\lfoot[\thepage]{} 		% gauche du pied de page
	\cfoot{} 			% milieu du pied de page
	\rfoot[]{\thepage} 		% droite du pied de page
	\fancyhead[RO, LE] {}	
	
\usepackage[acronym,xindy,toc,numberedsection,ucmark]{glossaries}
	% \newglossary[nlg]{notation}{not}{ntn}{Notation} % Création d'un type de glossaire 'notation'
	% \makeglossaries
	% \loadglsentries{appendix/glossary}			% Utilisation d'un fichier externe pour la définition des entrées (Glossaire.tex)				
\chapter*{Acknowledgments}
\addcontentsline{toc}{chapter}{Acknowledgments}

\begin{otherlanguage}{french}
\citationChap{
    L'important n’est pas ce qu’on a fait de nous mais ce que nous faisons nous-même de ce qu’on a fait de nous.
    }{\textit{Saint Genet, comédien et martyr}, Jean-Paul Sartre}
\end{otherlanguage}

The three years of this PhD have been a long and arduous, yet enriching, endeavour. In many regards, this period of my life has been similar to the \emph{Fool's Journey}, which I studied on a whim after an unexpected encounter with tarot card reading in the summer of 2023. Much like the Fool, who meets multiple influential figures on his road to enlightenment, I have met many wonderful people whom I feel obliged to mention here.\\

First of all, I address my warmest thanks to my two thesis supervisors, Fabian and Chloé. You have always been there for me when I needed you, especially during the difficult times; thank you for your unwavering support. I feel honored and blessed to have been allowed to pursue this PhD under your guidance.\\

Next, I would like to thank Claire Gardent and Smaranda Muresan, for having accepted to review this thesis, as well as Sophie Rosset, Benoît Favre, and Marine Carpuat, for having accepted to be part of my thesis committee. Thank you for having showed interest in my research work. I also want to thank Claire Gardent, again, and Oana Balalau, for having been part of my mid-term defence committee. Your positive and constructive feedback has meant a lot to me. Finally, I am very grateful to Pierre Colombo, my co-author, for having suggested to me the idea of my first article, which prompted me to reorient my research topic towards automatic evaluation. Thank you for your keen insight.\\

Now is perhaps a good time to mention all the people from Télécom Paris.\\

I address my sincere gratitude to the DIG team for the pleasant moments we shared together: the games of ``The Crew'' during lunch break, the DIG seminars and their iconic sushi orders (especially when I did get mine), the memorable DIG retreats, etc. I also want to thank the Social Computing group and more generally the people I met on the 5th floor during my rather long pilgrimage over there. Special thanks go to Simon C., for our many interesting discussions and our refreshing meetups in Paris; to Aina, for being my COLING partner and probably one of the gentlest people I've ever met; to Nils, for organizing the DIG seminars and simply for being a nice guy in general; to Jean-Louis, for introducing me to the academic world all those years back; to Antoine, for making me discover the wonders of French versification, to Zacchary and Thomas, for our intense and adrenaline-packed spikeball games, and to Rajaa, for always sharing Moroccan delicacies with us. And, of course, thanks to everyone else: Chadi, Lihu, Pierre-Henri, Émile, Gaël, Matthieu, Lucien, Arturo, Simon D., Marc J., Louis, Gabriel, Roman, Tiphaine, Yanzhu, Lorraine, Alisa, Maria, Samuel, Alicia, François, Tom, Weichen, Minh Huong, Hoang, Mehwish, Lanfang, Yiwen, Marc H., Mariam, Chaitanya, Nicoline, Fajrian, and all the others I may have missed.\\

\begin{otherlanguage}{french}

J'aimerais également remercier les personnes que j'ai rencontrées grâce au théâtre, en particulier au Studio Muller, et qui m'ont permis de vivre de très beaux moments pendant ces trois ans.\\

Merci à Lucas, mon grand frère retrouvé ; à Jocelyn D. et Roman, pour ce pique-nique des enfers ; à Emma C., pour cette saucisse volante et ces moments de sadisme ; à mes autres camarades d'Analyse-Action : David, Maëva, Esther, Victor R., Audrey, Margaux, Marion, Justin, Romain, Marie L., Julie, Valentine, Énola, Simon, Thibault, Emma M., Thomas, Charlotte H., pour m'avoir accompagné dans cette recherche de la vie ; à la compagnie B612 : Mélanie B., Nicky, Justine M., Juliette, Mélanie D., Quentin et Constance, pour ces écritures de plateau complètement farfelues ; à mes répliques de concours : Setti, Alexandre et Hélène ; à Katia et Vincent, pour ces deux années de prépa ; à Daniel, pour cette ouverture à la vie que tu m'as apportée ; à Jason, mon formidable professeur d'interprétation ; à Jocelyn Muller, pour nos entretiens toujours très riches et pour cette merveilleuse école que tu as fondée. Merci également à Charlène, Manon R., Yoann, Merlin, Florian S., Benjamin, Patrice, Charlotte B., Ingrid, Marie G., Grégoire, Virgile, Aristide, Mélissa O., Justine A., Anaïs, Erwan, et bien d'autres.\\

Je tiens aussi à remercier spécifiquement Randal Douc, à qui je dois très certainement d'avoir osé me lancer à la fois dans ce doctorat et dans le théâtre, et Mme Élisabeth Desrousseaux, pour m'avoir aidé à naviguer dans le brouillard lorsque je me sentais perdu.\\

À présent, j'aimerais remercier mes proches.\\

Merci à Sylvain, mon ami de toujours ; à Bruno, mon âme sœur ; à Victor P., mon chef d'état-major, à Jeanne, ma rencontre ; à ma coterie du Knight : Sébastien G., Sébastien C., Suzie et Sophia, pour nos fascinantes aventures ; à Lucas, Mathieu, Nicky, Gabin, David, Esther, Élise, Mélanie B., Merlin, Alexandre, Paul et Yaëlle, pour nos belles discussions ; à mes amis du lycée : Louis, Marc et Erwan ; à mes co de prépa : Loïc et Arnaud ; à mes camarades de Coët : Moïse, François, Alexis et Antonin ; à Amélie, ma partenaire de musées ; à Éléonore, pour nos agréables sorties ; à Adèle, pour cette enivrante soirée ; à Camille, pour cette douce après-midi ; enfin, à ceux que j'ai eu la chance d'apprendre à connaître : Antoine V.--L., Jocelyn D., Louis-Vincent et Léa, Kristina, Tristan, Jean-Baptiste, Stépan, Thibault, Ruben P., Frédérique et Florian L., Samuel, Tarek, Antoine D., Romain L., Côme, Théodore\ldots\\

Pour finir, je souhaite remercier ma famille.\\

Merci à Neak Mê, Pou Pin et Tata Anaïg, Pou Nin et Tata Pach, Mac Ming et Jean-Pierre. Merci à Om Srun et Mak Om, Om Kimly et Om Ny. Merci à mes cousins et cousines : Tana, Johan, Manon, Charlise, Matéo, Estelle, Alexandra, Amandine et Bang Kosal. Merci à mon frère Victor. Merci à ma mère, pour sa tendresse, et à mon père, pour son soutien.\\

À tous, je vous suis infiniment reconnaissant de tout ce que vous m'avez apporté.\\

Cela dit, pour paraphraser cette belle réplique de \textit{Racine carrée du verbe être}, de Wajdi Mouawad :

\begin{quote}
   `` Je ne vous aime pas parce que vous m'avez rendu meilleur ; je vous aime parce que je vous aime. ''
\end{quote}

\raggedleft
Vanves, le 3 décembre 2024.

\vspace*{0.5cm}

\begin{figure}[h!]
    \raggedleft
    \includegraphics[width=0.3\linewidth]{preamble/signature.png}
    \label{fig:signature}
\end{figure}

Cyril Chhun

\end{otherlanguage}

%Use \begin{otherlanguage}{french} ... \end{otherlanguage} for French